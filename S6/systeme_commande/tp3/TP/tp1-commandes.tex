\documentclass[a4paper]{article}

\usepackage[T1]{fontenc}
\usepackage[utf8]{inputenc}
\usepackage[french]{babel}
\usepackage{listings}

\usepackage{graphicx}
\usepackage{url}
\usepackage{ifthen}
\usepackage{keystroke}
\usepackage{fullpage}
\usepackage{hyperref}


%\title{\textbf{IMA 1$^{\mathaccent"7013 ere}$ année - T.P. Système}}
\title{\textbf{IS 3 - T.P. Système\\Interpréteur de commandes}}
%\date{Expressions rationnelles, grep, sort, sed, \ldots}
\date{}

\begin{document}
\vspace{-4cm}
\maketitle

\newboolean{reponse}
%\setboolean{reponse}{true}
\setboolean{reponse}{false}

Pour réaliser ce TP, vous n'utiliserez pas l'interface graphique mais
uniquement les terminaux en mode texte. Pour basculer de l'interface
graphique vers un terminal, utiliser la combinaison de touches \Ctrl +
\Alt + \keystroke{Fn}, où $1<=n<10$ (le premier terminal disponible
varie selon les machines, mais habituellement correspond à
\keystroke{F3}).

\section{Combinaison de commandes, wildcards}

\begin{enumerate}
\item Visualisez la liste des fichiers de votre répertoire d'accueil
  page par page. Affichez également les fichiers ``cachés'' (qui
  commencent par '.').
\item En utilisant une seule commande, \texttt{ls}, et sans changer de
  répertoire, listez tous les fichiers d'extension \texttt{.c} ou
  \texttt{.h} de votre répertoire d'Algorithmique et Programmation du
  S5. Utilisez un format d'affichage long et repérez la taille, date
  et heure de création, de ces fichiers.
\item Affichez uniquement les 15 premières lignes de l'aide de la
  fonction \texttt{scanf}.
\end{enumerate}

\section{Redirections}

\begin{enumerate}
\item Stockez l'aide de la commande \texttt{cat} dans le fichier
  \texttt{catAide.txt} ;
\item A l'aide d'un éditeur de texte (e.g. \texttt{nano},
  \texttt{emacs}, \texttt{vim}), éditez le fichier de nom
  \texttt{titreCat.txt} pour qu'il contienne le texte
  suivant~:
\begin{verbatim}
	----------------
	- COMMANDE CAT -
	- EXPLICATIONS -
	----------------
\end{verbatim}
\item Stockez dans le fichier \texttt{catTitreAide.txt} la
  con\emph{cat}énation des fichiers \texttt{titreCat.txt} et
  \texttt{catAide.txt} (cf \texttt{man cat}) ;    
\item Visualisez le fichier \texttt{strange.txt}. Comme ce
  dernier n'existe pas, que se passe-t-il ?
\item Relancez cette commande en redirigeant la sortie d'erreur vers
  le fichier \texttt{err.txt}, puis visualisez le contenu de ce dernier.
\item Relancez cette commande en redirigeant la sortie d'erreur vers
  \texttt{/dev/null}, puis visualisez le contenu de ce dernier. Pour
  comprendre le résultat, voir \texttt{man null}.
\item Écrivez le fichier suivant, par exemple avec l'éditeur
  \texttt{nano}, puis compilez et exécutez :
\begin{verbatim}
	#include <stdio.h>
	int main()
	{
		printf("message sur la sortie standard\n");
		fprintf(stderr,"message sur la sortie d'erreur\n");
		return 0;
	}
\end{verbatim}
  Lors de l'exécution, il n'y a pas de différence dans l'affichage des
  deux messages.  Redirigez maintenant la sortie d'erreur vers
  \texttt{/dev/null} et comparez l'affichage obtenu.
\end{enumerate}


\section{Filtres}

Vous allez travailler sur un fichier de journal d'impressions. Chaque
ligne a le format suivant~:

\texttt{date et heure | prénom.nom | nom\_machine | nom\_imprimante | nb\_pages}

Par exemple~:

\texttt{Thu Apr  6 10:30:23 CEST 2000|Vincent.Vandamme|hainaut.priv.eudil.fr|gayant|2}

\texttt{nb\_pages} est le nombre de pages effectivement imprimées ou
"aborted" ou "none" en cas d'échec à l'impression. Le fichier (un
extrait du vrai) se trouve dans
\texttt{/usr/localTP/tmpIMA/printaccounting}.  Ne le recopiez pas chez
vous car il est volumineux. Vous allez appliquer les commandes
ci-dessous sur ce fichier.

\begin{enumerate}
\item Donnez le nombre de lignes de ce fichier.

\ifthenelse{\boolean{reponse}}{
\texttt{wc -l /usr/local/tmpIMA/printaccounting}
}

\item Donnez le nombre de fois où l'impression est "aborted".

\ifthenelse{\boolean{reponse}}{
\texttt{grep "aborted" /usr/local/tmpIMA/printaccounting | wc -l}
}

\item Donnez le nombre de fois où l'impression est ni "aborted"
  ni "none".

\ifthenelse{\boolean{reponse}}{
\texttt{egrep -v "none|aborted" /usr/local/tmpIMA/printaccounting | wc -l}
}

\item Donnez les 10 plus grosses impressions. Attention, cela peut prendre du temps.

\ifthenelse{\boolean{reponse}}{
\texttt{cat printaccounting | sort -t{\textbackslash}| -k 5nr | head -10}
}

\item Transformez les lignes ``aborted'' selon l'exemple suivant.

\smallskip
  
\texttt{Thu Feb 3 11:54:31 CET 2000|Ahmed.Laib|gayant04|gayant|aborted}

\smallskip

devient

\smallskip

\texttt{Thu Feb 3 11:54:31 CET 2000|gayant|erreur}

\ifthenelse{\boolean{reponse}}{
\lstinline!cat /usr/local/tmpIMA/printaccounting | grep aborted | sed
's/\(.*\)|.*|.*|\(.*\)|.*/\\1|\\2|aborted/'!

ou

\lstinline!cat /usr/local/tmpIMA/printaccounting | grep aborted |cut-d'|' -f1,4 | sed "s/\(.*\)/\\1|aborted/"!
}
\end{enumerate}

\section{find}

\begin{enumerate}
\item Créez l'arborescence de fichiers et répertoires suivante :
  \begin{itemize}
  \item Dans le répertoire où vous lancez les commandes : un fichier
    \texttt{local.c}, un répertoire \texttt{ToSearch} ;
  \item Dans \texttt{ToSearch} : \texttt{Rep1}, \texttt{Rep2}, \texttt{niv1.c}
  \item Dans \texttt{Rep1} : \texttt{niv2-1.c}, \texttt{niv2-1.c\~}
  \item Dans \texttt{Rep2} : \texttt{niv2-2.c}, \texttt{niv2-2.c\~}, \texttt{niv2-3.c}
  \end{itemize}
\item Écrivez une commande qui permet d'afficher la liste de tous les
  fichiers c dans le répertoire courant et sa sous-arborescence. Essayez
  avec le répertoire contenant \texttt{ToSearch}.
\item Écrivez une commande qui permet d'afficher le contenu de tous les
  fichiers se terminant par \texttt{c\~} dans le répertoire courant et
  sa sous-arborescence. Essayez avec le répertoire contenant
  \texttt{ToSearch}, puis sur votre répertoire d'Algorithmique et Programmation du
  S5.
\item Affichez la liste des répertoires que vous avez créés dans les 7
  derniers jours. \textbf{NB} : on veut les répertoires et non pas les
  fichiers normaux qu'ils contiennent. Essayez sur votre répertoire
  personnel.
\end{enumerate}

\section{Bonus}

\begin{enumerate}
\item Écrivez une commande qui liste les 10 sous-répertoires de votre
  répertoire personnel dont le contenu est le plus volumineux. Indice
  : voir le man \texttt{du}, option \texttt{-d}.
  % du -d1 2>/dev/null | sort -nr | head -10
\item Créez une commande qui affiche la taille totale de l'ensemble
  des fichiers se terminant par \texttt{\~}, dans un répertoire donné
  et ses sous-répertoires. Indice : voir le man de \texttt{find}, en
  particulier les deux formes de l'option \texttt{exec}, et à nouveau
  le man de \texttt{du}.
  % find . -name "*~" -exec du -c {} +
\item Faites en sorte de pouvoir exécuter cette commande en tapant
  simplement \texttt{garbageSize}. Indice : cherchez \texttt{alias}
  dans le man de \texttt{bash}; pour chercher efficacement, consultez
  l'aide interactive de \texttt{man}.
\item Faites en sorte que la commande \texttt{garbageSize} soit
  utilisable pour tout nouveau terminal. Indice : voir
  \href{https://www.gnu.org/software/bash/manual/html_node/Bash-Startup-Files.html}{cette
    page}.
\end{enumerate}



\end{document}
